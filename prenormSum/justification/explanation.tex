\documentclass{article}
\usepackage[margin=2.5cm]{geometry}
\usepackage{amsmath}

\title{Linear modelling of size factors after summation}
\author{Aaron Lun}

\begin{document}
\maketitle

\section{Description of the algorithm}
Consider that each library $j$ has some bias $\theta_j$, representing the combined effects of capture efficiency and library size.
Further assume that there is some constantly expressed gene $i$, for which the expression rate $\lambda_i$ is the same across all cells.
The expectation of the count $y_{ij}$ for gene $i$ in cell $j$ is then defined as $\mu_{ij} = \theta_j\lambda_i$.

Assume that we have a set of cells $\mathcal{S}_k$.
Denote the sum of $y_{ij}$ across $\mathcal{S}_k$ as $s_{ik}$.
The expectation of $s_{ik}$ is
\[
    E(s_{ik}) = \lambda_i \sum_{j \in \mathcal{S}_k} \theta_j \;.
\]
Similarly, define $\mathcal{S}_0$ as the set of all cells in the data set, and $s_{i0}$ as the sum of counts across all cells for gene $i$.
The expectation of the ratio $r_{ik}$ of the count sums between $\mathcal{S}_k$ and $\mathcal{S}_0$ is defined as
\[
    E(r_{ik}) \approx \frac{E(s_{ik})}{E(s_{i0})} = \frac{\sum_{\mathcal{S}_k} \theta_j}{\sum_{\mathcal{S}_0} \theta_j} = \frac{\sum_{\mathcal{S}_k} \theta_j}{C}
\]
where $C$ is a constant that does not depend on the gene, cell or set $\mathcal{S}$.
The approximation is valid if the variance of the count sums is low, which should be the case for large sets.
This expectation is estimated by taking any robust average (e.g., median) of $r_{ik}$ across $i$.
Robustness is required to protect against DE genes with extreme ratios.
This approach assumes that most genes are non-DE -- see below for more details.

Anyway, the estimated value of $E(r_{ik})$ can be plugged in as the right-hand side of a linear equation involving $\theta_j$ as the terms being added.
Repeating this procedure for different sets of cells in $\mathcal{S}_{ik}$ will generate a system of linear equations.
This system can be solved with standard methods (e.g., QR decomposition) to obtain estimates of $\theta_j$ for all cells.
This approach may seem somewhat circuitous, given that $\theta_j$ could be estimated directly from the counts for each cell.
However, the idea is to use summation to avoid zeros and near-zero counts that are present in the single-cell counts.
These counts are problematic as they interfere with the calculation of median ratios and estimation of $\theta_j$.
Ratios computed from summed counts are more stable, and this stability will propagate back to the $\theta_j$ estimates per cell when the linear system is solved.

\section{Assumptions of this approach}
This approach requires some moderately strong assumptions regarding the nature of DE across the data set.
The use of the median is only valid when less than 50\% of genes are DE in any direction in each cell compared to the average across all cells,
    i.e., less than 50\% of genes can be upregulated and less than 50\% of genes can be downregulated.
This generally requires a proportion of genes to be constantly expressed across all cells in the data set (otherwise all genes would be DE against the average in at least one cell),
    and is only guaranteed to be fulfilled when that proportion is equal to or greater than 50\% of all genes.

Another sticking point with this approach is that, in theory, it is possible to obtain negative estimates for $\theta_j$.
Such values are obviously ridiculous.
It seems that generation of such values would require samples with very small library sizes relative to the other cells, 
    such that the true value of $\theta_j$ is already close to zero.
Stochastic error may then be sufficient to push the estimate below zero.
In general, this scenario seems somewhat pathological, as samples with small library sizes should be removed prior to analysis.

\section{Selecting cells to sum}
A key step of this approach lies in how to select cells for each set $\mathcal{S}_{ik}$.
Ideally, one would prefer to sum cells with similar expression profiles.
This ensures that the same set of genes are DE against the average for all cells in the set,
    which minimizes the total amount of DE in the set and reduces the chance of violating the non-DE assumption.
However, any attempt to quantify similarity between cells would require normalization.
This becomes circular -- we need similarity to normalize, but we also need normalization to compute similarities.

The quick-and-dirty solution is to use the library size as a rough proxy for cell similarity.
This is motivated by the fact that different cell types tend to have systematic differences in the library sizes, e.g., due to differences in total RNA or capture efficiency.
First, cells are ordered by their total counts and partitioned into two groups, depending on whether the ranking of that cell is odd or even.
The cells are then arranged in a ring, with odd cells on the left and even cells on the right.
Conceptually, one can imagine starting at the 12 o'clock position on the ring, at the largest libraries; moving clockwise through the even cells for decreasing library size;
    hitting the smallest libraries at 6 o'clock; and then, continuing to move clockwise, now through the odd cells, for increasing library size.
For summation, a sliding window is moved cell-by-cell across this ring.
Each window contains the same number of cells, where all cells of the ring lying in the window are used to define a single instance of the set $\mathcal{S}_{ik}$.
Thus, each window defines a separate equation in the linear system.
The use of a ring means that the window is still fully defined at the smallest and largest libraries.

The total number of equations in the linear system is equal to the number of cells.
Each cell is represented in $w$ equations where $w$ refers to the size of the window.
Additional equations can be added to improve the precision of the estimates, by using different values of $w$.
However, note that this is only possible when $w$ is coprime with the number of cells.
Otherwise, the columns of the matrix become linearly dependent such that no solution is possible.
Coprime identification is not easy, so instead, we add pseudo-cells with averaged counts to ensure that the total number of cells is not divisible by 2 or 5.
Values of $w$ are then set to 20, 25, 40, 50, 80 and 100.
These are large enough to obtain stable sums yet small enough to maintain resolution.
This increases the number of equations and ensures that each cell is represented in over 300 equations.

\end{document}
